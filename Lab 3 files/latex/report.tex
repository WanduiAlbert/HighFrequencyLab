\documentclass[twocolumn, aps, apl]{revtex4-1}

\usepackage{graphicx}
\usepackage{fancyhdr}
\usepackage{amsmath}
\usepackage{caption}
\usepackage{enumerate}
\usepackage{subcaption}
\usepackage{textcomp}
\usepackage{placeins}
\usepackage{blindtext}

\begin{document}
\title{Lab 3. Low Noise Amplifiers }
\author{Albert Wandui}
\affiliation{EE 152: High Frequency Systems Lab.}
\maketitle

\section*{Introduction}\label{sec:introduction}
The goal of this lab was to design, assemble and test the noise performance of a low noise amplifier (LNA). We optimized our LNA for gain, noise performance and output matching. There are a number of stages involved in the design of a well performing LNA. First, we selected the DC bias points of the amplifier by a suitable choice of resistors at the base and collector of the Bipolar Junction Transistor (BJT). Once the bias point was established, we designed the input matching network for the amplifier.

The input matching network is chosen to minimize the noise power in the amplifier. To see this, we consider the noise factor of the amplifier, F, which is the ratio of the signal to noise ratio at the output to that at the input. The noise factor captures the added noise due to the amplifier that is present in the output. For an amplifier with an effective temperature $T_e$,  and a reference temperature $T_0 = 290 K$ (room temperature), the noise factor is given by the equation

\begin{equation}
    F = 1 + \frac{T_e}{T_0}.
\end{equation} 

The noise in the amplifier can be characterized using three parameters, $T_{min}, N \textrm{and} \Gamma_{opt}$. $T_{min}$, is the minimum achievable noise temperature for the transistor design. N is a scaling factor that weights the noise contribution due to the mismatch between the amplifier and the input network. Lastly, $\Gamma_{opt}$ is the reflection coefficient of the source that minimizes the noise in the system. Using these three parameters, we can express the effective temperature of the amplifier as 

\begin{equation}
    T_e = T_{min} + 4 N T_0 \frac{|\Gamma_s - \Gamma_{opt}|^2}{\left(1 - |\Gamma_s|^2\right)\left(1 - |\Gamma_{opt}|^2\right)}.
\end{equation}

Thus we designed our input matching network to present $\Gamma_s = \Gamma_{opt}$ to the transistor biasing network at the input of the amplifier. Usually, the MWO simulations give the optimal reflection coefficient looking into the transistor network thus we matched based on the condition, $\Gamma_s = \Gamma_{in}^*$. 

We also payed close attention to stability considerations. At a given frequency, we wanted our amplifier to be unconditionally stable. That is, for any given choice of passive source and load impedances, $|\Gamma_{in}| < 1, |\Gamma_{out} < 1|$. We can then define criteria $\mu_1 > 1, \mu_2 > 1$ based on this condition to ensure that we will always be unconditionally matched. Defining $\Delta = S_{11}S_{22} - S_{12}S_{21}$, we obtain

\begin{align}
    \mu_1 &= \frac{1 - |S_{11}|^2}{|S_{11} - S_{22}^* \Delta| + |S_{21}S_{12}|} \\
    \mu_2 &= \frac{1 - |S_{22}|^2}{|S_{22} - S_{11}^* \Delta| + |S_{21}S_{12}|}
\end{align}

If either, $\mu_1 > 1$ or $\mu_2 > 1$, then the network is unconditionally stable at the given frequency. To achieve stability, we added shunt inductors at the input and output of the network to provide high impedance to the input. In addition, we tuned the series resistor at the output to ensure good stability over the entire frequency range of interest albeit at some cost to the gain.

With the input matching network designed, fixing $\Gamma_s $ using $ \Gamma_{opt}$ fixes $\Gamma_{out}$ since $\Gamma_{out} = S_{22} + \frac{S_{21}S_{12}\Gamma_{opt}}{1 - \Gamma_{opt} S_{11}}$. Thus we designed the output matching network to achieve this condition. 

With these steps in the design of the matching network, we were able to achieve low noise performance at the desired frequency of 4 GHz.

\section*{LNA Design}\label{sec:design}
Based on the steps discussed in the introduction, we first checked $T_{min}$ for each bias condition for the transistor to select the one that gave the best noise performance. We selected the 2V 6 mA bias BJT because it had the lowest $T_{min} = 47.24 K$ at 4 GHz.

Once the selection was made, we designed the bias network, choosing $R_B = 45 k \Omega$ and $R_c = 10 \Omega$ as shown in figure \ref{fig:dcnet}. The transistor .s2p file specified a bias voltage of 2V at output and a bias current of 5.96 mA with $V_BE = 0.7914 V, I_B = 0.0178 mA$. Using 15 $\Omega$ at the collector gives $V_{cc} = 2.1 V$ and a bias resistance of $R_B = 45 k \Omega$. We placed large bias resistors close to $V_{cc}$ instead of next to the input of the transistor to prevent resonances between the resistor and the inductors in the network. This differs from the ideal case where we would place the shunt bias resistors as close as possible to the input to minimize the noise.

    \begin{figure*}[!htbp]
        \centering
        \includegraphics[scale=0.35]{dc_net.png}
        \caption{Schematic showing the bias resistors chosen to achieve a 6mA collector current and a 2V Bias in the design. The bypass capacitors are also shown.}
        \label{fig:dcnet}
    \end{figure*}

In addition, we changed all the microstrip widths to 21 mils and used the lengths specified in the lab guide. We then added 5.6 nH inductors to the base and collector to improve stability. This was done while ensuring that we did not change $T_{min}$ significantly by doing so. Using the .s2p files for the inductors, we checked to ensure that there were no self-resonances in the region of interest. Had great stability above 2.5 GHz. See figure \ref{fig:LNAnet}, \ref{fig:stability}

    \begin{figure*}[!htbp]
        \centering
        \includegraphics[scale=0.35]{LNA_net.png}
        \caption{Schematic showing the high impedance inductors used to stabilize the amplifier response. Small 10$\Omega$ resistors were also added in series to enhance stability.}
        \label{fig:LNAnet}
    \end{figure*}

    \begin{figure}[!htbp]
    \centering
    \includegraphics[scale=0.3]{stability.png}
    \caption{A plot of the stability parameters, $\mu_1, \mu_2$ of the network. We have good stability in region where either $\mu_1 > 1$ or $\mu_2 > 1$. }
    \label{fig:stability}
    \end{figure}

The input matching network was then designed, by choosing $\Gamma_s = \Gamma_{opt}$ as determined for our choice of bias and stability network for the amplifier. The schematic shown in figure \ref{fig:inputnet} shows our choice of lumped element series L and shunt C for this matching network.

    \begin{figure*}[!htbp]
    \centering
    \includegraphics[scale=0.35]{input_net.png}
    \caption{Input matching network for the amplifier.}
    \label{fig:inputnet}
    \end{figure*}

Designed the output matching network based on the value of $\Gamma_{in} = \Gamma_{opt}$ which fixes the value of $\Gamma_l$. The output matching network is shown in figure \ref{fig:outputnet}. The small resistor at the output of the amplifier was set to $30 \Omega$ for stability of the network at a slight cost to the gain of the system.

    \begin{figure*}[!htbp]
    \centering
    \includegraphics[scale=0.35]{output_net.png}
    \caption{Output matching network for the amplifier.}
    \label{fig:outputnet}
    \end{figure*}

Design noise temperature $T_e = 58.35 K$ at 4 GHz. Only slightly above the minimum noise temperature for the choice of transistor.

\FloatBarrier

\section*{Lab Testing}

\begin{itemize}
    \item populated the board with all the necessary capacitors, inductors, connectors and resistors as well as the BJT. Soldered all the components onto the board. Added a 10pF DC block to the RF input of the network because the Noise Figure Analyzer is not ac coupled. 

    \item Calibrated the VNA at -30.0 dBm input power and measured the S parameters of the amplifier circuit. Setup see figure \ref{fig:ampimg}. OVP at 2.50V and OCP at 0.01 A. See figure \ref{fig:ampSparams}.

    \begin{figure}[!htbp]
    \centering
    \includegraphics[scale=0.05]{LNA_1.jpg}
    \caption{FieldFox setup for S parameter measurements on the low noise amplifier.}
    \label{fig:ampimg}
    \end{figure}

    \begin{figure}[!htbp]
    \centering
    \includegraphics[scale=0.3]{amp_S_params.png}
    \caption{Measured S parameters of the amplifier at -30.0 dBm input power.}
    \label{fig:ampSparams}
    \end{figure}

    \item Recalibrated the VNA to a -5.0 dBm input power level for better measurements of the filter response. Setup see figure \ref{fig:filterimg}. Measured the S parameters of the filter. See figure \ref{fig:filterSparams}.

    \begin{figure}[!htbp]
    \centering
    \includegraphics[scale=0.05]{Filter.jpg}
    \caption{Measurement setup for the S parameters of the Low Pass Filter.}
    \label{fig:filterimg}
    \end{figure}

    \begin{figure}[!htbp]
    \centering
    \includegraphics[scale=0.35]{filter_S_params.png}
    \caption{Measured S parameters of the filter at -5.0 dBm input power.}
    \label{fig:filterSparams}
    \end{figure}


\end{itemize}

\FloatBarrier
\subsection{Noise Measurements}

\begin{itemize}
    \item for the Y factor measurements, used a Noise Figure Analyzer (NFA) calibrated to a room temperature (290 K) noise source.

    \item plugged the noise source to the input of the amplifier and the output port of the amplifier to the NFA. Same biasing at 2V and 6mA as shown in figure \ref{fig:ampnoiseimg}.

    \begin{figure}[!htbp]
    \centering
    \includegraphics[scale=0.05]{LNA_Noise.jpg}
    \caption{Noise measurement setup using the NFA, noise source and the amplifier}
    \label{fig:ampnoiseimg}
    \end{figure}

    \item swept through frequency range twice and saved the noise and gain data. Effective temperature, $T_e = 128.668 K$, a factor of about 2.2 times higher than the expected noise levels. A plot of the equivalent input noise temperature, as simulated vs. measured shown in figure \ref{fig:amptempcomparison}.

    \begin{figure}[!htbp]
    \centering
    \includegraphics[scale=0.3]{LNA_noisetemp_comparison.pdf}
    \caption{A comparison of the measured vs. simulated noise performance of the amplifier. The high excess noise above 1 GHz is clearly visible.}
    \label{fig:amptempcomparison}
    \end{figure}

    \item Hooked up the low pass filter to the input of the amplifier and retook the measurements. As expected, the noise temperature was much higher at $T_e = 196.025 K$. With the filter at the output, the noise temperature was lower at $T_e = 128.81 K$. A comparison of the two measurements is shown in figure \ref{fig:filtertempcomparison}.

    \begin{figure}[!htbp]
    \centering
    \includegraphics[scale=0.3]{Filter_noisetemp_comparison.pdf}
    \caption{A comparison of the noise performance of the amplifier with the filter at the input (behind) and filter at the output (ahead) of the LNA.}
    \label{fig:filtertempcomparison}
    \end{figure}

\end{itemize}

\section*{Analysis}

\begin{itemize}
    \item The simulated and measured noise temperature of the LNA showed strong deviations \ref{fig:amptempcomparison}. The shape of the noise temperature profile as a function of the frequency was the same, but elevated at higher frequencies. Given more time, I would have attempted to account for noise contributions from the passives in the network as well as from the power supply. Maybe something else but need ideas!!!

    \item The change in noise temperature between the filter at input and filter at output cases can be modelled well and with good accuracy. The Friis formula captures how to calculate the effective noise factor of a system based on the noise factors of the constituent subsystems. Given $F_i$ as the noise factor for the $i$th subsystem and $G_i$ as the gain of that subsystem, we can state the Friis formula as

    \begin{equation}
        F_{total} - 1 = (F_1 - 1) + \frac{F_2 - 1}{G_1} + \frac{F_3 - 1}{G_1 \times G_2} + \ldots.  
    \end{equation} 

    Specializing it for our case, we note that for the amplifier, $F - 1 = \frac{T_e}{T_0}$ where $T_e$ is the equivalent input noise temperature of the amplifier as measured and $T_0 = 290 K$. For the low pass filter, following the discussion in the introductory section \ref{sec:introduction}, the noise factor is given by the relation $F - 1 = L - 1$, where L is the attenuation of the filter ($L = 1/G$).

    For the filter at the input of the amplifier, then the total effective temperature is given by

    \begin{equation}
        T_{TOT, behind} = T_{LNA} + \frac{L - 1}{G_{LNA}} \times T_0.
    \end{equation}

    Using this equation, I have plotted a comparison between the measured and expected effective noise temperature as shown in figure \ref{fig:behindnoisetemp}
    
    \begin{figure}[!htbp]
    \centering
    \includegraphics[scale=0.3]{Filter_behind_noisetemp.pdf}
    \caption{A comparison of expected and measured noise performance of the amplifier with the filter at the input (behind) of the LNA.}
    \label{fig:behindnoisetemp}
    \end{figure}

    On the other hand, when the filter is placed at the output of the amplifier, then the total effective temperature is given by

    \begin{equation}
        T_{TOT, ahead} = \left(L - 1 \right) \times T_0 + L \times T_{LNA}.
    \end{equation}

    Similarly, we can compare the predicted effective temperature from this equation with the actual measured performance as shown in figure \ref{fig:aheadnoisetemp}.
    \begin{figure}[!htbp]
    \centering
    \includegraphics[scale=0.3]{Filter_ahead_noisetemp.pdf}
    \caption{A comparison of expected and measured noise performance of the amplifier with the filter at the output (ahead) of the LNA.}
    \label{fig:aheadnoisetemp}
    \end{figure}

    A comparison of these two equations, immediately shows why we have a higher effective temperature when the filter is at the output. Since the filter is a passive device, its gain is always less than 1 and as a result, $L > 1$. If the amplifier has a high enough gain, then the gain factor in the denominator makes that term very small and thus for the filter behind case, $T_{TOT} \approx T_{LNA}$. For the filter ahead case, we instead have $T_{TOT} \approx L \times T_{LNA}$ which would be bigger since L is greater than 1.


    \item Simulations of the amplifier circuit with the filter at the input show an expected equivalent temperature of 100.2 K up from about 60.86 K. This deviation is partly due to the mismatch at the input introduced by the filter. In the filter band, the filter has a very small reflection coefficient. Talk about this some more.

    \item Adding shunt capacitors and series inductors at the ports to model parasitics gives a slightly better match in the frequency range 1.5 - 4.5 GHz. In addition, slight changes to the values of the matching network help create a better match between simulation and measured behavior. However, a good match could not be found over the entire frequency range of interest. Improvements at low frequencies led to a much worse match at high frequencies. We can thus account well for the behavior close to the design frequency of 4 GHz but not at much higher frequencies. See \ref{fig:changedampsims}

    \begin{figure}[!htbp]
    \centering
    \includegraphics[scale=0.4]{LNA_simulated_measured.png}
    \caption{A comparison of expected and simulated amplifier response chosen to better match with the measured response.}
    \label{fig:changedampsims}
    \end{figure}

\end{itemize}

\section*{Conclusions}

This lab gave me good experience with the design considerations in making a good low noise amplifier. I have a better appreciation of the difficulties inherent in identifying all possible noise sources to try and achieve some desired noise specifications. There was also plenty of lab experience in soldering and hopefully the time put in now will pay off in future labs and projects. 
\end{document}