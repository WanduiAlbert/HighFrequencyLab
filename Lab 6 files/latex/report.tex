% \documentclass[onecolumn, aps, floatfix]{revtex4-1}
\documentclass{article}

\usepackage[margin=0.75in]{geometry}
\usepackage{graphicx}
\usepackage{fancyhdr}
\usepackage{amssymb}
\usepackage{amsmath}
\usepackage{caption}
\usepackage{subcaption}
\usepackage{enumerate}
\usepackage{subcaption}
\usepackage{textcomp}
\usepackage{placeins}
\usepackage{blindtext}


\begin{document}

\title{Lab 6. Oscillator Design and Antenna Measurements }
\author{Albert Wandui \\
\textit{EE 152: High Frequency Systems Lab.}}
\maketitle

% {\let\newpage\relax\maketitle} % Ensures that there is no page break after maketitle

\section*{Introduction}\label{sec:introduction}
The goal of this lab is to design and fabricate a microwave oscillator. In addition, we measured the transmission of a cantenna system and calculated the gain of the antenna. Microwave oscillators convert dc power to RF and are ubiquitous in microwave systems as sources of power. 

We realized our oscillator using negative resistance active devices. These are devices that present a negative input impedance. While positive resistance devices dissipate power, negative resistance devices supply power. Such negative resistance devices can be realized by making a transistor unstable as is further discussed in section \ref{sec:oscdesign}.
 - Give the negative oscillator feedback equations.

 - Talk about the antenna equations
\section*{Oscillator Design}\label{sec:oscdesign}
\begin{itemize}
    \item Base biasing and instability

    \item Convertion from 2 port to 3 port measurements

    \item Check the stability at the emitter and collector. Choose the emitter as the output and the collector as the load

    \item Design the matching network on the output to transform 50 Ohms to the unstable region.

    \item Add the RL circuit at the collector and the collector biasing network

    \item check the input impedance of the circuit for resonances
\end{itemize}


\section*{Oscillator Measurements}\label{sec:ampdesign}

\begin{itemize}
    \item Setup the FFox to SA mode. Bias the oscillator to 2.5 V, 20 mA. Check for oscillations. Change the value of the inductor to get as close as possible to 5.9 GHz.
\end{itemize}

\section*{Antenna Measurements}\label{sec:testing}

\begin{itemize}
    \item Setup the cantennas 1m apart. Line up the cantennas. Set the Hittite power to 10.0 dBm. Measure the power output at the receiver antenna.
\end{itemize}

\section*{Analysis}\label{sec:analysis}

\begin{itemize}
    \item Table of oscillator frequency, power and resistor and inductor values

    \item How does the power change with tuning?

    \item Plot of the final oscillator spectrum. dBc values of the second and third harmonics.

    \item How could we oscillate close to the goal of 5.9 GHz.

    \item Calculate the gain of the cantenna.

\end{itemize}

\section*{Conclusions}


\end{document}